\documentclass[addpoints]{exam}

\usepackage{mathtools}
\usepackage{amsmath}
\usepackage{amssymb}
\usepackage{geometry}
\usepackage{venndiagram}
\usepackage{graphicx}
\usepackage{arydshln}
\makeatletter
  \renewcommand*\env@matrix[1][*\c@MaxMatrixCols c]{%
    \hskip -\arraycolsep
    \let\@ifnextchar\new@ifnextchar
  \array{#1}}
\makeatother

% Header and footer.
\pagestyle{headandfoot}
\runningheadrule
\runningfootrule
\runningheader{Homework 5}{Linear Algebra}{}
\runningfooter{}{Page \thepage\ of \numpages}{}
\firstpageheader{}{}{}

\boxedpoints
\printanswers
\qformat{} %Comment this to number questions, uncomment this to not number questions

\newcommand\union\cup
\newcommand\inter\cap

\title{Linear Algebra Spring 23\\ Homework 5}
\author{Ali Muhammad Asad}

\begin{document}
\maketitle
\begin{sloppypar}
\section*{\textbf{Chapter 2: Determinants and Matrix Properties}}
\subsection*{\textbf{Ex Set 2.3: Properties of the Determinant Function}}
% \vspace{1mm}
\begin{questions}
    \question
    \textbf{Question 3} By inspection, explain why det$(A) = 0$
    $$ A = \begin{bmatrix}
        -2 & 8 & 1 & 4 \\ 3 & 2 & 5 & 1 \\ 1 & 10 & 6 & 5 \\ 4 & -6 & 4 & -3
    \end{bmatrix} $$
    \begin{solution}
        By inspection, the following row operation can be done \\ 
        $ R_2 \rightarrow R_2 + R_1 \implies A = \begin{bmatrix}
            -2 & 8 & 1 & 4 \\ 1 & 10 & 6 & 5 \\ 1 & 10 & 6 & 5 \\ 4 & -6 & 4 & -3
        \end{bmatrix} $
        \\ We can clearly see that there are two identical rows: \\ 
        $ R_2 = \rightarrow R_2 - R_3 \implies A = \begin{bmatrix}
            -2 & 8 & 1 & 4 \\ 0 & 0 & 0 & 0 \\ 1 & 10 & 6 & 5 \\ 4 & -6 & 4 & -3
        \end{bmatrix} $
        \\ By using EROS, we get a row of zeroes, therefore, det$(A) = 0$.
    \end{solution}
    \pagebreak
    \question
    \textbf{Question 4} Use Theorem 2.3.3 to determine which of the following matrices are invertible. 
        
        (a) $ \begin{bmatrix}
            1 & 0 & -1 \\ 9 & -1 & 4 \\ 8 & 9 & -1
        \end{bmatrix} $
        \hspace*{5mm} (b) $ \begin{bmatrix}
            4 & 2 & 8 \\ -2 & 1 & -4 \\ 3 & 1 & 6
        \end{bmatrix} $
        \hspace*{5mm} (c) $ \begin{bmatrix}
            \sqrt{2} & -\sqrt{7} & 0 \\ 3\sqrt{2} & -3\sqrt{7} & 0 \\ 5 & -9 & 0 
        \end{bmatrix} $
        \hspace*{5mm} (d) $ \begin{bmatrix}
            -3 & 0 & 1 \\ 5 & 0 & 6 \\ 8 & 0 & 3
        \end{bmatrix} $
        \begin{solution}

            Theorem 2.3.3: \textit{A square matrix is invertible if and only if det$ (A) \neq 0 $}

            Further, suppose each matrix to be as $A$
            \begin{parts}
                \part det$ (A) = 1\begin{bmatrix}
                    -1 & 4 \\ 9 & -1
                \end{bmatrix} - 0 -1\begin{bmatrix}
                    9 & -1 \\ 8 & 9
                \end{bmatrix} = 1(1 - 36) + 0 -1(81 + 8) \neq 0$

                Therefore is invertible.

                % \part It can be clearly seen that 
                \part det$ (A) = 4\begin{bmatrix}
                    1 & -4 \\ 1 & 6
                \end{bmatrix} + -2\begin{bmatrix}
                    -2 & -4 \\ 3 & 6
                \end{bmatrix} + 8 \begin{bmatrix}
                    -2 & 1 \\ 3 & 1
                \end{bmatrix} = 4(6 + 4) - 2(-12 + 12) + 8(-2 - 3) = 40 - 40 = 0$
                $ \therefore det(A) = 0 $

                Also, it can be clearly seen that $ R_1 $ is a scalar multiple of $ R_2 $. Therefore this matrix is not invertible.

                \part The last column is a column of zeroes, therefore the determinant is zero. Hence this matrix is not invertible.

                \part Column of zeroes, determinant is zero, therefore not invertible.
            \end{parts}
        \end{solution}

        \question
        \textbf{Question 5} Let $ A = \begin{bmatrix}
            a & b & c \\ d & e & f \\ g & h & i
        \end{bmatrix} $. Assuming that det$(A) = -7$, find: 

        (a) det$(3A)$ \hspace*{5mm} (b) det$(A^{-1})$ \hspace*{5mm} (c) det$(2A^{-1})$ \hspace*{5mm} (d) det$((2A)^{-1})$ \hspace*{5mm} (e) det$\begin{bmatrix}
            a & g & d \\ b & h & e \\ c & i & f
        \end{bmatrix}$
        \begin{solution}
            By a basic property, det$ (kA) = k^n $det$ (A) $
            \begin{parts}
                \part $ det(3A) = 3^3 det(A) = 27 * -7 = -189 $
                \part $ det(A^{-1}) = \frac{1}{det(A)} = \frac{1}{-7} $
                \part $ det(2A^{-1}) = 2^3 det(A^{-1}) = \frac{8}{det(A)} = -\frac{8}{7} $
                \part $ det((2A)^{-1}) = \displaystyle\frac{1}{det(2A)} = \frac{1}{2^3 det(A)} = -\frac{1}{8 * 7} = -\frac{1}{56}$
                \part From inspection, if we take the tranpose of the matrix, we get $ \begin{bmatrix}
                    a & b & c \\ g & h & i \\ d & e & f
                \end{bmatrix} $. Then by swapping $ R_2 $ with $ R_3 $, we get $ \begin{bmatrix}
                    a & b & c \\ d & e & f \\ g & h & i
                \end{bmatrix} $ which is the same as the matrix given in the question. Therefore $ det = -(-7) $ as we took the tranpose. So $ det = 7 $.
            \end{parts}
        \end{solution}

        \question
        \textbf{Question 6} Without directly evaluating, show that $ x = 0 $ and $ x = 2 $ satisfy $ \begin{vmatrix}
            x^2 & x & 2 \\ 2 & 1 & 1 \\ 0 & 0 & -5
        \end{vmatrix} = 0 $
        \begin{solution}
            
            (1) $x = 0$:
            $ \begin{vmatrix}
                0 & 0 & 2 \\ 2 & 1 & 1 \\ 0 & 0 & -5
            \end{vmatrix} \implies R_3 \rightarrow R_3 + \frac{5}{2}R_1 \begin{vmatrix}
                0 & 0 & 2 \\ 2 & 1 & 1 \\ 0 & 0 & 0
            \end{vmatrix}$ \\ 
            Since we have a row of zeroes, the determinant is also 0. Hence $ x = 0 $ satisfies the matrix.

            (2) $ x = 2 $
            $ \begin{vmatrix}
                4 & 2 & 2 \\ 2 & 1 & 1 \\ 0 & 0 & -5
            \end{vmatrix} \implies R_1 \rightarrow \frac{1}{2}R_1 \begin{vmatrix}
                2 & 1 & 1 \\ 2 & 1 & 1 \\ 0 & 0 & -5
            \end{vmatrix}$ \\ 
            Since we have two identical rows, the determinant of the matrix is 0. Hence shown that $ x = 2 $ satisfies the matrix. 
        \end{solution}

        \question
        \textbf{Question 7} Without directly evaluating show that $ det \begin{bmatrix}
            b+c & c+a & b+a \\ a & b & c \\ 1 & 1 & 1 
        \end{bmatrix} = 0$
        \begin{solution}
            
            By inspection, $ R_1 \rightarrow R_1 + R_2 \implies \begin{bmatrix}
                a+b+c & a+b+c & a+b+c \\ a & b & c \\ 1 & 1 & 1
            \end{bmatrix} $ \\ Then $ R_1 \rightarrow \frac{1}{a + b + c} \implies \begin{bmatrix}
                1 & 1 & 1 \\ a & b & c \\ 1 & 1 & 1
            \end{bmatrix} $

            Hence we are left with two identical matrices. Therefore the determinant is zero.
        \end{solution}
        \pagebreak
        \question
        \textbf{Question 8} Prove the identity without evaluating the determinants \\ 
        $ \begin{vmatrix}
            a_1 & b_1 & a_1 + b_1 + c_1 \\ 
            a_2 & b_2 & a_2 + b_2 + c_2 \\ 
            a_3 & b_3 & a_3 + b_3 + c_3
        \end{vmatrix} = \begin{vmatrix}
            a_1 & b_1 & c_1 \\
            a_2 & b_2 & c_2 \\ 
            a_3 & b_3 & c_3 
        \end{vmatrix}$
        \begin{solution}
            Column operations have no effect on the determinant: \\
            LHS: 
            $ C_3 \rightarrow C_3 - C_2 - C_1 \implies \begin{vmatrix}
                a_1 & b_1 & c_1 \\
                a_2 & b_2 & c_2 \\ 
                a_3 & b_3 & c_3 
            \end{vmatrix} = $ RHS. Hence proved.
        \end{solution}
        
        \question
        \textbf{Question 9} Prove the identity without evaluating determinants. \\ 
        $ \begin{vmatrix}
            a_1 + b_1 & a_1 - b_1 & c_1 \\ 
            a_2 + b_2 & a_2 - b_2 & c_2 \\
            a_3 + b_3 & a_3 - b_3 & c_3
        \end{vmatrix} = -2 \begin{vmatrix}
            a_1 & b_1 & c_1 \\
            a_2 & b_2 & c_2 \\ 
            a_3 & b_3 & c_3
        \end{vmatrix}$
        \begin{solution}
            LHS: $ C_1 \rightarrow C_1 + C_2 \implies \begin{vmatrix}
                2a_1 & a_1 - b_1 & c_1 \\
                2a_2 & a_2 - b_2 & c_2 \\ 
                2a_3 & a_3 - b_3 & c_3  
            \end{vmatrix} $ \\ 
            We can take 2 common from the first column, and then apply the column operations: $ C_2 \rightarrow C_2 - C_1 \implies 2\begin{vmatrix}
                a_1 & -b_1 & c_1 \\
                a_2 & -b_2 & c_2 \\ 
                a_3 & -b_3 & c_3  
            \end{vmatrix} = $ RHS \\ 
            Further taking -1 common from the second column, we get $ -2\begin{vmatrix}
                a_1 & b_1 & c_1 \\
                a_2 & b_2 & c_2 \\ 
                a_3 & b_3 & c_3
            \end{vmatrix} $ Hence proved.
        \end{solution}

        \question
        \textbf{Question 11} Prove the identity without evaluating determinants. \\ 
        $ \begin{vmatrix}
            a_1 & b_1 + ta_1 & c_1 + rb_1 + sa_1 \\ 
            a_2 & b_2 + ta_2 & c_2 + rb_2 + sa_2 \\ 
            a_3 & b_3 + ta_3 & c_3 + rb_3 + sa_3 
        \end{vmatrix} = \begin{vmatrix}
            a_1 & a_2 & a_3 \\ b_1 & b_2 & b_3 \\ c_1 & c_2 & c_3
        \end{vmatrix}$
        \begin{solution}
            LHS: $ C_2 \rightarrow C_2 - tC_1, c_3 \rightarrow C_3 - sC_1 \implies \begin{vmatrix}
                a_1 & b_1 & c_1 + rb_1 \\ 
                a_2 & b_2 & c_2 + rb_2 \\ 
                a_3 & b_3 & c_3 + rb_3 
            \end{vmatrix} $ \\ 
            $ C_3 \rightarrow C_3 - rC_2 \implies \begin{vmatrix}
                a_1 & b_1 & c_1 \\ 
                a_2 & b_2 & c_2 \\
                a_3 & b_3 & c_3 
            \end{vmatrix} $ by tranpose $ \implies \begin{vmatrix}
                a_1 & a_2 & a_3 \\ b_1 & b_2 & b_3 \\ c_1 & c_2 & c_3
            \end{vmatrix} $ [$det(A) = det(A^T)$] \\ 
            Hence proved.
        \end{solution}

        \question
        \textbf{Question 13} Use Theorem 2.3.3 to show that $ \begin{bmatrix}
            \sin^2\alpha & \sin^2\beta & \sin^2\gamma \\ 
            \cos^2\alpha & \cos^2\beta & \cos^2\gamma \\ 
            1 & 1 & 1 
        \end{bmatrix} $ is not invertible for any values of $ \alpha, \beta, \text{ and } \gamma $. 
        \begin{solution}
            By $ R_1 \rightarrow R_1 + R_2 $ we will get $ \sin^2\alpha + \cos^2\alpha, \sin^2\beta + \cos^2\beta, \sin^2\gamma + \cos^2\gamma $ in the first row. And by basic trigonometric identity, it will reduce to 1. So the first row becomes a row of 1s, and so does the third row. Since we have two identical rows, the determinant is 0. Therefore, the matrix is not invertible for any values of $ \alpha, \beta, \gamma $.
        \end{solution}

        \question
        \textbf{Question 16} Let $A$ and $B$ be $ n_{\times}n $ matrices. Show that if $A$ is invertible, then $ det(B) = det(A^{-1}BA) $.
        \begin{solution}
            If $A$ is invertible, then $ det(A) \neq 0 \therefore det(A^{-1}) = \frac{1}{det(A)} $ is defined. \\ 
            $ det(B) = det(A^{-1}) * det(B) * det(A) $ by the theorem. \\ $ \implies det(B) = \frac{1}{det(A)} * det(B) * det(A) $ \\ 
            $ \implies det(B) = det(B) $ Hence proved.   
        \end{solution}

        \question
        \textbf{Question 21} Let $A$ and $B$ be $n_{\times}n$ matrices. You know from earlier work that $AB$ is invertible if $A$ and $B$ are invertible. What can you say about the invertibility of $AB$ if one or both of the factors are singular? Explain your reasoning. 
        \begin{solution}
            If $A$ and $B$ are $n_{\times}n$ matrices, then $ det(AB) = det(A) * det(B) $. \\ 
            If either one of the matrices are not invertible, then either $ det(A) = 0 $ or $ det(B) = 0 $ or both equal to 0. \\ 
            Then $ det(AB) = det(A) * det(B) \implies det(AB) = 0 $. \\ 
            Then $AB$ is also not invertible. So if one or both of the factors are singular, then $AB$ is not invertible.
        \end{solution}

\end{questions}
\end{sloppypar}
\end{document}
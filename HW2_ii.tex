\documentclass[addpoints]{exam}

\usepackage{amsmath}
\usepackage{amssymb}
\usepackage{geometry}
\usepackage{venndiagram}
\usepackage{graphicx}
\usepackage{mleftright}

% Header and footer.
\pagestyle{headandfoot}
\runningheadrule
\runningfootrule
\runningheader{HW1}{Linear Algebra}{}
\runningfooter{}{Page \thepage\ of \numpages}{}
\firstpageheader{}{}{}

\boxedpoints
\printanswers
\qformat{} %Comment this to number questions, uncomment this to not number questions

\newcommand\union\cup
\newcommand\inter\cap

\title{Linear Algebra\\ Homework 2 part ii}
\author{Ali Muhammad Asad}

\begin{document}
\maketitle
\section*{\textbf{Chapter 1 : Linear Equations and Matrices}}
\subsection*{\textbf{Ex Set 1.5 Elementary Matrices and a Method for finding $ A^{-1} $}}
% \vspace{1mm}
\begin{questions}
    \question
    \textbf{2. } Find a row operation that will restore the given matrix to an identity matrix \\
    (a) $ \begin{bmatrix}
        1 & 0 \\ -3 & 1
    \end{bmatrix} $ \hspace{5mm} (b) $ \begin{bmatrix}
        1 & 0 & 0 \\ 0 & 1 & 0 \\ 0 & 0 & 3
    \end{bmatrix} $ \hspace{5mm} (c) $ \begin{bmatrix}
        0 & 0 & 0 & 1 \\ 0 & 1 & 0 & 0 \\ 0 & 0 & 1 & 0 \\ 1 & 0 & 0 & 0
    \end{bmatrix} $ \hspace{5mm} (d) $ \begin{bmatrix}
        1 & 0 & -\frac{1}{7} & 0 \\ 0 & 1 & 0 & 0 \\ 0 & 0 & 1 & 0 \\ 0 & 0 & 0 & 1
    \end{bmatrix} $ 
    \begin{solution}
        
        (a) $ R_3 \longrightarrow R_3 + 3R_1 $ \\ 
        (b) $ R_3 \longrightarrow \frac{1}{3} R_3 $ \\ 
        (c) $ R_3 \longleftrightarrow R_1 $ \\
        (d) $ R_1 \longrightarrow R_1 + \frac{1}{7}R_3 $

    \end{solution}

    \textbf{3. } Consider the matrices: 
    $$ A = \begin{bmatrix}
        3 & 4 & 1 \\ 2 & -7 & -1 \\ 8 & 1 & 5
    \end{bmatrix}, \;\;\;\; B = \begin{bmatrix}
        8 & 1 & 5 \\ 2 & -7 & -1 \\ 3 & 4 & 1 
    \end{bmatrix}, \;\;\;\; C = \begin{bmatrix}
        3 & 4 & 1 \\ 2 & -7 & -1 \\ 2 & -7 & 3
    \end{bmatrix} $$

    Find the elementary matrices $ E_1, E_2, E_3, \text{ and } E_4 $ such that \\ 
    (a) $ E_1A = B $ \hspace{5mm} (b) $ E_2B = A $ \hspace{5mm} (c) $ E_3A = C $ \hspace{5mm} (d) $ E_4C = A $
    \begin{solution}
        
        (a) $A$ can be converted to $B$ with the following row operation: $ R_1 \longleftrightarrow R_3 $. \\ 
        $ \implies E_1 = \begin{bmatrix}
            0 & 0 & 1 \\ 0 & 1 & 0 \\ 1 & 0 & 0
        \end{bmatrix} $

        (b) $ B $ can be converted to $A$ with the following row operation: $ R_1 \longleftrightarrow R_3 $. \\ 
        $ \implies E_2 = \begin{bmatrix}
            0 & 0 & 1 \\ 0 & 1 & 0 \\ 1 & 0 & 0
        \end{bmatrix} $

        (c) $A$ can be converted to $C$ with the following row operation: $ R_3 \longrightarrow R_3 - 2R_1 $. \\
        $ \implies E_3 = \begin{bmatrix}
            1 & 0 & 0 \\ 0 & 1 & 0 \\ -2 & 0 & 1
        \end{bmatrix} $ 

        (d) $C$ can be converted to $A$ with the following row operation: $ R_3 \longrightarrow R_3 + 2 R_1 $. \\ 
        $ \implies E_4 = \begin{bmatrix}
            1 & 0 & 0 \\ 0 & 1 & 0 \\ 2 & 0 & 1
        \end{bmatrix} $
    \end{solution}

    \textbf{9. } Find the inverse of the following $ 4 \times 4 $ matrices, where $ k_1, k_2, k_3, k_4 $ and $k$ are all nonzero \\ 
    (a) $ \begin{bmatrix}
        k_1 & 0 & 0 & 0 \\ 0 & k_2 & 0 & 0 \\ 0 & 0 & k_3 & 0 \\ 0 & 0 & 0 & k_4
    \end{bmatrix} $ \hspace{5mm} (b) $ \begin{bmatrix}
        0 & 0 & 0 & k_1 \\ 0 & 0 & k_2 & 0 \\ 0 & k_3 & 0 & 0 \\ k_4 & 0 & 0 & 0
    \end{bmatrix} $
    \begin{solution}
        
        (a) Using row operations, we can obtain the inverse by: \\
        $ \renewcommand\arraystretch{1.3}
    \mleft[ 
    \begin{array}{cccc|cccc}
        k_1 & 0 & 0 & 0 & 1 & 0 & 0 & 0 \\ 
        0 & k_2 & 0 & 0 & 0 & 1 & 0 & 0 \\ 
        0 & 0 & k_3 & 0 & 0 & 0 & 1 & 0 \\ 
        0 & 0 & 0 & k_4 & 0 & 0 & 0 & 1
    \end{array}
    \mright] $ \\ 
    The following operations give us the inverse: \\ 
    $ R_1 \rightarrow \frac{1}{k_1}R_1 $ \hspace{5mm} $ R_2 \rightarrow \frac{1}{k_2}R_2 $ \hspace{5mm} $ R_3 \rightarrow \frac{1}{k_3} $ \hspace{5mm} $ R_4 \rightarrow \frac{1}{k_4}R_4 $ \\ 
    $ \implies 
    \renewcommand\arraystretch{1.3}
    \mleft[ 
    \begin{array}{cccc|cccc}
        1 & 0 & 0 & 0 & \frac{1}{k_1} & 0 & 0 & 0 \\ 
        0 & 1 & 0 & 0 & 0 & \frac{1}{k_2} & 0 & 0 \\
        0 & 0 & 1 & 0 & 0 & 0 & \frac{1}{k_3} & 0 \\ 
        0 & 0 & 0 & 1 & 0 & 0 & 0 & \frac{1}{k_4}
    \end{array}
    \mright]
    $

    Then our inverse is $ \begin{bmatrix}
        \frac{1}{k_1} & 0 & 0 & 0 \\ 0 & \frac{1}{k_2} & 0 & 0 \\ 0 & 0 & \frac{1}{k_3} & 0 \\ 0 & 0 & 0 & \frac{1}{k_4}
    \end{bmatrix} $

    \vspace{5mm}
    (b) Using row operations, we can obtain the inverse by: \\
    $ \renewcommand\arraystretch{1.3}
    \mleft[
    \begin{array}{cccc|cccc}
        0 & 0 & 0 & k_1 & 1 & 0 & 0 & 0 \\ 
        0 & 0 & k_2 & 0 & 0 & 1 & 0 & 0 \\ 
        0 & k_3 & 0 & 0 & 0 & 0 & 1 & 0 \\ 
        k_4 & 0 & 0 & 0 & 0 & 0 & 0 & 1
    \end{array}
    \mright]  
    $ \\ 
    First we swap the rows: $ R_1 \longleftrightarrow R_4 $ and $ R_2 \longleftrightarrow R_3 $. \\  
    $ \implies 
    \renewcommand\arraystretch{1.3}
    \mleft[
    \begin{array}{cccc|cccc}
        k_4 & 0 & 0 & 0 & 0 & 0 & 0 & 1 \\ 
        0 & k_3 & 0 & 0 & 0 & 0 & 1 & 0 \\ 
        0 & 0 & k_2 & 0 & 0 & 1 & 0 & 0 \\ 
        0 & 0 & 0 & k_1 & 1 & 0 & 0 & 0
    \end{array}
    \mright] 
    $ \\ 
    Next we can perform the following row operations, $ R_1 \rightarrow \frac{1}{k_4}R_1, \; R_2 \rightarrow \frac{1}{k_3}R_2, \; R_3 \rightarrow \frac{1}{k_2}R_3, \; R_4 \rightarrow \frac{1}{k_1}R_4 \\ 
    \implies 
    \renewcommand\arraystretch{1.3}
    \mleft[
    \begin{array}{cccc|cccc}
        1 & 0 & 0 & 0 & 0 & 0 & 0 & \frac{1}{k_4} \\ 
        0 & 1 & 0 & 0 & 0 & 0 & \frac{1}{k_3} & 0 \\ 
        0 & 0 & 1 & 0 & 0 & \frac{1}{k_2} & 0 & 0 \\ 
        0 & 0 & 0 & 1 & \frac{1}{k_1} & 0 & 0 & 0
    \end{array}
    \mright]
    $

    Then our inverse is $ \begin{bmatrix}
        0 & 0 & 0 & \frac{1}{k_4} \\ 
        0 & 0 & \frac{1}{k_3} & 0 \\
        0 & \frac{1}{k_2} & 0 & 0 \\
        \frac{1}{k_1} & 0 & 0 & 0
    \end{bmatrix} $
    \end{solution}

    \textbf{10.} Consider the matrix $ A = \begin{bmatrix}
        1 & 0 \\ -5 & 2
    \end{bmatrix} $ \\ 
    (a) Find the elementary matrices $ E_1 $ and $ E_2 $ such that $ E_2E_1A = I $ \\ 
    (b) Write $ A^{-1} $ as a product of two elementary matrices. \\ 
    (c) Write $A$ as a product of two elementary matrices.
    \begin{solution}
        
        (a) To reduce $A$ to identity, we can observe the following row operations: $ R_2 \rightarrow R_2 + 5 R_1 $ followed by $ R_2 \rightarrow \frac{1}{2}R_2 $. Then our elementary matrices in order become \\ 
        $ E_1 = \begin{bmatrix}
            1 & 0 \\ 5 & 1
        \end{bmatrix}, E_2 = \begin{bmatrix}
            1 & 0 \\ 0 & \frac{1}{2}
        \end{bmatrix} $ as first $A$ will get premultiplied by $ E_1 $ and then followed by premultiplication with $E_2$.

        (b) Since $ A^{-1}A = I $ holds true, and $A$ is reduced to an identity matrix following the two row operations, the matrix $ A^{-1} $ can be defined as $ A^{-1} = E_2E_1 $.

        (c) Similarly, $ E_2E_1A = I \implies E_1^{-1}E_2^{-1}E_2E_1A = E_1^{-1}E_2^{-1}I \implies A = E_1^{-1}E_2^{-1}$   
    \end{solution}

    \textbf{16. } Show that $ A = \begin{bmatrix}
        0 & a & 0 & 0 & 0 \\ b & 0 & c & 0 & 0 \\ 0 & d & 0 & e & 0 \\ 0 & 0 & f & 0 & g \\ 0 & 0 & 0 & h & 0
    \end{bmatrix} $ is not invertible for any values of the entries. 
    
    [\textit{Hint: The row containing $b$ must become the first row after ERO have been carried out and the one with $g$ must be  the last row. Why? this caused problems when trying to reduce the middle column to only one nonzero entry.}]
    \begin{solution}
        It is sufficient to show that $A$ can have a row of zeroes through elementary row operations to show that $A$ is not invertible. \\ 
        We can perform the following row operations: $ R_1 \rightarrow \frac{d}{a}R_1 $ and $ R_5 \rightarrow \frac{e}{h}R_5 $. \\ 
        $ \implies A = \begin{bmatrix}
        0 & d & 0 & 0 & 0 \\ b & 0 & c & 0 & 0 \\ 0 & d & 0 & e & 0 \\ 0 & 0 & f & 0 & g \\ 0 & 0 & 0 & e & 0
    \end{bmatrix} $ \\ 
    Then by the following row operations, $ R_3 \rightarrow R_3 - R_1 $ and $ R_3 \rightarrow R_3 - R_5 $ we get \\ 
    $ A = \begin{bmatrix}
        0 & d & 0 & 0 & 0 \\ b & 0 & c & 0 & 0 \\ 0 & 0 & 0 & 0 & 0 \\ 0 & 0 & f & 0 & g \\ 0 & 0 & 0 & e & 0
    \end{bmatrix} $ \\ 
    Therefore, we are left with a row of zeroes which shows that $A$ is not invertible for any values of the entries as we get a row of zeroes through elementary row operations.

    \end{solution}

\end{questions}

\end{document}

% $ \renewcommand\arraystretch{1.3}
%     \mleft[
%     \begin{array}{cccc|cccc}

%     \end{array}
%     \mright]  
%     $
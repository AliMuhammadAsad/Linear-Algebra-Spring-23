\documentclass[addpoints]{exam}

\usepackage{amsmath}
\usepackage{amssymb}
\usepackage{geometry}
\usepackage{venndiagram}
\usepackage{graphicx}

% Header and footer.
\pagestyle{headandfoot}
\runningheadrule
\runningfootrule
\runningheader{HW1}{Linear Algebra}{}
\runningfooter{}{Page \thepage\ of \numpages}{}
\firstpageheader{}{}{}

\boxedpoints
\printanswers
\qformat{} %Comment this to number questions, uncomment this to not number questions

\newcommand\union\cup
\newcommand\inter\cap

\title{Linear Algebra Spring 23\\ Proofs}
\author{Ali Muhammad Asad}

\begin{document}
\maketitle
\begin{sloppypar}
\section*{\textbf{Chapter 1 : Linear Equations and Matrices}}
% \subsection*{\textbf{Ex Set 1.4 : Inverses; Rules of Matrix Arithmetic}}
% \vspace{1mm}
\begin{questions}
    \question
    \textbf{1. } If $ A_1, A_2,..., A_n $ are invertible matrices of the same size, then $ (A_1A_2...A_n)^{-1} = A_{n}^{-1}A_{n-1}^{-1}...A_{2}^{-1}A_{1}^{-1} $. [Prove using induction]
    \begin{solution}
        
    \end{solution}
    
    \question 
    \textbf{2. } (a) If $A$ is an ivnertible matrix, then $ (A^{-1})^{-1} = A $ 

    \hspace{5.5mm} (b) Prove that $ (A^n)^{-1} = (A^{-1})^n $ for $ n = 0, 1, 2, ... $ [Prove using induction] 
    
    [Hint: $ A^n = AA...A $ n times]
    \begin{solution}
        
    \end{solution}

    \question
    \textbf{3. } (a) If $ A, B $ are matrices s.t. $AB$ is defined, then $ (AB)^T = B^TA^T $. 

    (b) Prove that the transpose of a product of any number of matrices is equal to the product of their transposes in the reverse order i.e. $ (A_1A_2...A_n)^T = A_n^T...A_2^TA_1^T $ [Prove using induction]
    \begin{solution}
        
    \end{solution}

    \question
    \textbf{4. } If a system of equations has a unique solution, then Gaussian Elimination will find it. [\textit{Hint: Think induction - if we can prove it for one variable, and then prove it for $k$ variables, then we can prove it for $ (k+1) $ variables.}]
    \begin{solution}
        
    \end{solution}

    \question
    \textbf{5. } Prove that the reduced row echelon form is always unique [Qns needs to be worded properly, will do that]
    \begin{solution}
        
    \end{solution}
    
    

\end{questions}
\end{sloppypar}
\end{document}
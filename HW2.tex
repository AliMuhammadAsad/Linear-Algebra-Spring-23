\documentclass[addpoints]{exam}

\usepackage{amsmath}
\usepackage{amssymb}
\usepackage{geometry}
\usepackage{venndiagram}
\usepackage{graphicx}
\usepackage{mleftright}

% Header and footer.
\pagestyle{headandfoot}
\runningheadrule
\runningfootrule
\runningheader{HW2}{Linear Algebra}{}
\runningfooter{}{Page \thepage\ of \numpages}{}
\firstpageheader{}{}{}

\boxedpoints
\printanswers
\qformat{} %Comment this to number questions, uncomment this to not number questions

\newcommand\union\cup
\newcommand\inter\cap
\makeatletter
\renewcommand*\env@matrix[1][*\c@MaxMatrixCols c]{%
  \hskip -\arraycolsep
  \let\@ifnextchar\new@ifnextchar
  \array{#1}}
\makeatother

\title{Linear Algebra\\ Homework 2}
\author{Ali Muhammad Asad}

\begin{document}
\maketitle
\begin{sloppypar}
\section*{\textbf{Chapter 1 : Linear Equations and Matrices}}
\subsection*{\textbf{Ex Set 1.3 Matrices and Matrix Operations}}
% \vspace{1mm}
\begin{questions}
    \question
    \textbf{1.} $$ A \text{\hspace{15mm}} B \text{\hspace{15mm}} C \text{\hspace{15mm}} D \text{\hspace{15mm}} E$$
    $$ \text{\hspace{1mm}} (4 \text{ x } 5) \text{\hspace{7mm}} (4 \text{ x } 5) \text{\hspace{7mm}} (5 \text{ x } 2) \text{\hspace{8mm}} (4 \text{ x } 2) \text{\hspace{8mm}} (5 \text{ x } 4) $$
    
    Determine which of the following matrix operations are defined. For those that are defined, give the size of the resulting matrix. 

    (a) $ BA $ \\ (b) $ AC + D $ \\ (c) $ AE + B $ \\ (d) $ AB + B $ \\ (e) $ E(A + B) $ \\ (f) $ E(AC) $
    \begin{solution}
        
        (a) Not defined since $B$ has 5 columns but $A$ has 4 rows so columns $\neq$ rows. \\ 
        (b) Is defined as $AC$ is defined and will result in a (4 x 2) matrix which has the same order as $D$ and can be added. \\ 
        (c) Not defined. $AE$ will result in a matrix of order (4 x 4) which is not the same as B which has order (4 x 5). So addition is not possible so not defined. \\ 
        (d) Not defined as $AB$ is not defined as $A$ has 5 columns while $B$ has 4 rows. \\ 
        (e) Is defined and will result in a matrix of order (5 x 5) as $A + B$ will give a matrix of order (4 x 5) which when is post multiplied by $E$ will give matrix of order (5 x 5). \\ 
        (f) Is defined. $AC$ will give a (4 x 2) matrix which when post multiplied by $E$ will give a (5 x 2) matrix. 
    \end{solution}
    
    \question
    % \textbf{2. } Solve the following matrix equation for $ a, b, c, \text{ and } d.$ \\ $\begin{bmatrix}
    %     a-b & b + c \\ 
    %     3d + c & 2a - 4d
    % \end{bmatrix} = \begin{bmatrix}
    %     8 & 1 \\ 
    %     7 & 6
    % \end{bmatrix}$
    % \begin{solution}
    %     Our equations are as follows: \\ 
    %     (1) $ a - b = 8 \implies a = 8 + b \implies b = a -  $ \\ 
    %     (2) $ b + c = 1 \implies b = 1 - c \implies c = 1 - b $ \\ 
    %     (3) $ 3d + c = 7 $ \\ 
    %     (4) $ 2a - 4d = 6 $
        
    %     Substituting (2) in (3), we get $ 3d + 1 - b = 7 \implies -b + 3d = 6 $ (5) \\ 
    %     Substituting (1) in (4), we get $ 16 + 2b - 4d = 6 \implies 2b - 4d = -10 $ (6) 

    %     Solving (5) and (6) simultaneously, $ d = 1, b = - 3 $. Then plugging in the values of $d$ and $b$ in (1) and (2), $ a = 5 $ and $ c = 4 $. \\ 
    %     $ a = 5, b = -3, c = 4, d = 1 $
    % \end{solution}

    \question
    \textbf{7. } Let \newline\newline
    $A = \begin{bmatrix}
        3 & -2 & 7 \\ 
        6 & 5 & 4 \\ 
        0 & 4 & 9
    \end{bmatrix} \text{ and } 
    B = \begin{bmatrix}
        6 & -2 & 4 \\ 
        0 & 1 & 3 \\ 
        7 & 7 & 5
    \end{bmatrix}$  

    Use the method of Example 7 [given in the book] to find \\ 
    (a) the first row of $AB$ \\ 
    (b) the third row of $AB$ \\ 
    (c) the second column of $AB$ \\ 
    (d) the first column of $BA$ \\ 
    (e) the third row of $AA$ \\ 
    (f) the third column of $AA$ 
    \begin{solution}
        
        (a) The first row of $AB$ can be obtained by \\ 
        $ \begin{bmatrix}
            3 & -2 & 7
        \end{bmatrix} \begin{bmatrix}
            6 & -2 & 4 \\ 
            0 & 1 & 3 \\ 
            7 & 7 & 5
        \end{bmatrix} = \begin{bmatrix}
            3(6)-2(0)+7(7) & 3(-2)-2(1)+7(7) & 3(4)-2(3)+7(5) 
        \end{bmatrix}$ \\ 
        $ = \begin{bmatrix}
            67 & 41 & 41
        \end{bmatrix} $

        (b) the third row of $AB$ can be obtained by \\ 
        $ \begin{bmatrix}
            0 & 4 & 9
        \end{bmatrix} \begin{bmatrix}
            6 & -2 & 4 \\ 
            0 & 1 & 3 \\ 
            7 & 7 & 5
        \end{bmatrix} = \begin{bmatrix}
            0(6)+4(0)+9(7) & 0(-2)+4(1)+9(7) & 0(4)+4(3)+9(5)
        \end{bmatrix}$ \\ 
        $ \begin{bmatrix}
            63 & 67 & 57
        \end{bmatrix} $

        (c) the second column of $AB$ can be obtained by \\ 
        $ \begin{bmatrix}
            3 & -2 & 7 \\ 
            6 & 5 & 4 \\ 
            0 & 4 & 9
        \end{bmatrix} \begin{bmatrix}
            -2 \\ 1 \\ 7
        \end{bmatrix} = \begin{bmatrix}
            41 \\ 21 \\ 67
        \end{bmatrix} $

        (d) the first column of $BA$ can be obtained by \\ 
        $ \begin{bmatrix}
            6 & -2 & 4 \\ 
            0 & 1 & 3 \\ 
            7 & 7 & 5
        \end{bmatrix} \begin{bmatrix}
            3 \\ 6 \\ 0
        \end{bmatrix} = \begin{bmatrix}
            6 \\ 6 \\ 63
        \end{bmatrix}$

        (e) the third row of $AA$ \\ 
        $ \begin{bmatrix}
            0 & 4 & 9
        \end{bmatrix} \begin{bmatrix}
            3 & -2 & 7 \\ 
            6 & 5 & 4 \\ 
            0 & 4 & 9
        \end{bmatrix} = \begin{bmatrix}
            24 & 56 & 97
        \end{bmatrix} $

        (f) the third column of $AA$ \\ 
        $ \begin{bmatrix}
            3 & -2 & 7 \\ 
            6 & 5 & 4 \\ 
            0 & 4 & 9
        \end{bmatrix} \begin{bmatrix}
            7 \\ 4 \\ 9
        \end{bmatrix} = \begin{bmatrix}
            76 \\ 98 \\ 97
        \end{bmatrix}$
    \end{solution}

    \question
    \textbf{8. } Let $A$ and $B$ be matrices from Q7. Use method of Example 9 [from the book] to \\ 
    (a) express each column matrix of $AB$ as a linear combination of the column matrices of $A$ \\ 
    (b) express each column matrix of $BA$ as a linear combination of the column matrices of $B$
    \begin{solution}
        $ AB = \begin{bmatrix}
            67 & 41 & 41 \\ 
            64 & 21 & 59 \\ 
            63 & 67 & 57
        \end{bmatrix} $ and $ BA = \begin{bmatrix}
            6 & -6 & 70 \\ 
            6 & 17 & 31 \\ 
            63 & 41 & 122
        \end{bmatrix} $

        (a) $ \begin{bmatrix}
            67 \\ 64 \\ 63
        \end{bmatrix} = 6\begin{bmatrix}
            3 \\ 6 \\ 0
        \end{bmatrix} + 0\begin{bmatrix}
            -2 \\ 5 \\ 4
        \end{bmatrix} + 7\begin{bmatrix}
            7 \\ 4 \\ 9
        \end{bmatrix}$ \\ 
        $ \begin{bmatrix}
            41 \\ 21 \\ 67
        \end{bmatrix} = -2\begin{bmatrix}
            3 \\ 6 \\ 0
        \end{bmatrix} + 1\begin{bmatrix}
            -2 \\ 5 \\ 4
        \end{bmatrix} + 7\begin{bmatrix}
            7 \\ 4 \\ 9
        \end{bmatrix}$ \\ 
        $ \begin{bmatrix}
            41 \\ 59 \\ 57
        \end{bmatrix} = 4\begin{bmatrix}
            3 \\ 6 \\ 0
        \end{bmatrix} + 3\begin{bmatrix}
            -2 \\ 5 \\ 4
        \end{bmatrix} + 5\begin{bmatrix}
            7 \\ 4 \\ 9
        \end{bmatrix} $

        (b) $ \begin{bmatrix}
            6 \\ 6 \\ 63
        \end{bmatrix} = 3\begin{bmatrix}
            6 \\ 0 \\ 7
        \end{bmatrix} + 6\begin{bmatrix}
            -2 \\ 1 \\ 7
        \end{bmatrix} + 0\begin{bmatrix}
            4 \\ 3 \\ 5
        \end{bmatrix}$ \\ 
        $ \begin{bmatrix}
            -6 \\ 17 \\ 41
        \end{bmatrix} = -2\begin{bmatrix}
            6 \\ 0 \\ 7
        \end{bmatrix} + 5\begin{bmatrix}
            -2 \\ 1 \\ 7
        \end{bmatrix} + 4\begin{bmatrix}
            4 \\ 3 \\ 5
        \end{bmatrix}$ \\ 
        $ \begin{bmatrix}
            70 \\ 31 \\ 122
        \end{bmatrix} = 7\begin{bmatrix}
            6 \\ 0 \\ 7
        \end{bmatrix} + 4\begin{bmatrix}
            -2 \\ 1 \\ 7
        \end{bmatrix} + 9\begin{bmatrix}
            4 \\ 3 \\ 5
        \end{bmatrix}$
    \end{solution}

    \question
    \textbf{12. } (a) Show that if $AB$ and $BA$ are both defined, then $AB$ and $BA$ are square matrices. 

    \hspace{7.5mm} (b) Show that if $A$ is an $m$x$n$ matrix and $A(BA)$ is defined, then $B$ is an $n$x$m$ matrix.
    \begin{solution}
        
        (a) Let $A$ be an $m$x$n$ matrix, and let $B$ be an $k$x$l$ matrix. Then for $AB$ to be defined, the number of columns of $A$ have to be equal to the number of rows of $B$ so $n = k$. 

        Similarly, for $BA$ to be defined, number of columns of $B$ have to be equal to the number of rows of $A$ so $m = l$. Then $A$ is an $m$x$n$ matrix and $B$ is an $n$x$m$ matrix. 
        
        Then $AB$ is an $m$x$m$ matrix and $BA$ is an $n$x$n$ matrix which are both square matrices. Hence shown. 

        (b) Let $A$ be an $m$x$n$ matrix. If $A(BA)$ is defined, then the number of columns of $A$ has to be equal to the number of rows of $BA$, then $BA$ has $n$ number of rows. \\ 
        For $BA$ to be defined, the number of columns of $B$ has to be equal to the number of rows of $A$. Hence, $B$ has $m$ columns. To produce matrix $BA$ with $n$ number of rows, $B$ must have $n$ number of rows. \\ 
        Hence shown that $B$ is an $n$x$m$ matrix.
    \end{solution}

    \question
    \textbf{13. } In each part, find matrices $A, x,$ and $b$ that express the given system of linear equations as a single matrix equation $ Ax = b $. \\ 
    (a) 
    
    $ \begin{matrix}
        2x_1 &- 3x_2 & +5x_3 & = 7 \\ 
        9x_1 & -x_2 & +x_3 & = -1 \\ 
        x_1 & +5x_2 & +4x_3 & = 0
    \end{matrix} $

    (b) 
    
    $ \begin{matrix}
        4x_1 & & -3x_3 & +x_4 = 1 \\ 
        5x_1 & +x_2 & & -8x_4 = 3 \\ 
        2x_1 & -5x_2 & +9x_3 & -x_4 & = 0 \\ 
        & 3x_2 & -x_3 & +7x_4 & = 2
    \end{matrix} $
    \begin{solution}

        (a)
        $ A = \begin{bmatrix}
            2 & -3 & 5 \\ 
            9 & -1 & 1 \\ 
            1 & 5 & 4
        \end{bmatrix}, x = \begin{bmatrix}
            x_1 \\ x_2 \\ x_3
        \end{bmatrix}, b = \begin{bmatrix}
            7 \\ -1 \\ 0
        \end{bmatrix} $

        (b) $ A = \begin{bmatrix}
            4 & 0 & -3 & 1 \\ 
            5 & 1 & 0 & -8 \\ 
            2 & -5 & 9 & -1 \\ 
            0 & 3 & -1 & 7
        \end{bmatrix}, b = \begin{bmatrix}
            x_1 \\ x_2 \\ x_3 \\ x_4
        \end{bmatrix}, b = \begin{bmatrix}
            1 \\ 3 \\ 0 \\ 2
        \end{bmatrix} $
    \end{solution}

    \question
    \textbf{17. } In each part, determine whether block multiplication can be used to compute $AB$ from the given partitions. If so, compute the product by block multiplication. [\textbf{\textit{Note}} See Exercise 15 of book.]

    % (a) $$ A = \begin{bmatrix}[ccc|c]
    %     -1 & 2 & 1 & 5 \\ 

    % \end{bmatrix} $$
    (a) \[ A = \renewcommand\arraystretch{1.3}
    \mleft[ 
    \begin{array}{ccc|c}
        -1 & 2 & 1 & 5 \\ 
        0 & -3 & 4 & 2 \\ 
        \hline
        1 & 5 & 6 & 1
    \end{array}
    \mright], B = \renewcommand\arraystretch{1.3}
    \mleft[
    \begin{array}{cc|c}
        2 & 1 & 4 \\ 
        -3 & 5 & 2 \\ 
        \hline 
        7 & -1 & 5 \\ 
        0 & 3 & -3
    \end{array}
    \mright] \]
    
    (b) \[ A = \renewcommand\arraystretch{1.3}
    \mleft[ 
    \begin{array}{cccc}
        -1 & 2 & 1 & 5 \\ 
        0 & -3 & 4 & 2 \\ 
        \hline
        1 & 5 & 6 & 1
    \end{array}
    \mright], B = \renewcommand\arraystretch{1.3}
    \mleft[
    \begin{array}{c|c|c}
        2 & 1 & 4 \\ 
        -3 & 5 & 2 \\  
        7 & -1 & 5 \\ 
        0 & 3 & -3
    \end{array}
    \mright] \]
    \begin{solution}
        
        (a) The matrix $AB$ can not be computed using block multiplication as the partitioned matrix $A_{11}$ is an $ 2 \text{x} 3 $ matrix whereas the partitioned $B_{11}$ is an $ 2 \text{x} 2 $ matrix. So matrix multiplication is not possible as the number of columns of $A_{11}$ are not equal to the number of rows of $B_{11}$.

        (b) The matrix $ A_{11} $ is an $2$x$4$ matrix, $A_{21}$ is an $1$x$4$ matrix, and  $B_{11}, B_{12}, \text{ and } B_{13}$ are all $4$x$1$ matrices. So matrix multiplication is possible between $A_{11}$ and all partitions of $B$, and $A_{21}$ and all partitions of $B$. \\ 
        The resultant matrix should be a $3$x$3$ matrix with 6 partitions as follows:\[ A = \renewcommand\arraystretch{1.3}
        \mleft[ 
        \begin{array}{c|c|c}
            A_{11}B_{11} & A_{11}B_{12} & A_{11}B_{13} \\
            \hline 
            A_{21}B_{11} & A_{21}B_{12} & A_{21}B_{13}
        \end{array}
        \mright]\]

        \[ A_{11}B_{11} = \renewcommand\arraystretch{1.3}
        \mleft[ 
        \begin{array}{cccc}
            -1 & 2 & 1 & 5 \\ 
            0 & -3 & 4 & 2
        \end{array}
        \mright] \renewcommand\arraystretch{1.3}
        \mleft[
        \begin{array}{c}
            2 \\ 3 \\ 7 \\ 0
        \end{array}
        \mright] = \begin{bmatrix}
            -1 \\ 37
        \end{bmatrix}\]

        \[ A_{11}B_{12} = \renewcommand\arraystretch{1.3}
        \mleft[ 
        \begin{array}{cccc}
            -1 & 2 & 1 & 5 \\ 
            0 & -3 & 4 & 2
        \end{array}
        \mright] \renewcommand\arraystretch{1.3}
        \mleft[
        \begin{array}{c}
            1 \\ 5 \\ -1 \\ 3
        \end{array}
        \mright] = \begin{bmatrix}
            23 \\ -13 
        \end{bmatrix}\]

        \[ A_{11}B_{13} = \renewcommand\arraystretch{1.3}
        \mleft[ 
        \begin{array}{cccc}
            -1 & 2 & 1 & 5 \\ 
            0 & -3 & 4 & 2
        \end{array}
        \mright] \renewcommand\arraystretch{1.3}
        \mleft[
        \begin{array}{c}
            4 \\ 2 \\ 5 \\ -3
        \end{array}
        \mright] = \begin{bmatrix}
            -10 \\ 8
        \end{bmatrix}\]

        \[ A_{21}B_{11} = \renewcommand\arraystretch{1.3}
        \mleft[ 
        \begin{array}{cccc}
            1 & 5 & 6 & 1
        \end{array}
        \mright] \renewcommand\arraystretch{1.3}
        \mleft[
        \begin{array}{c}
            2 \\ 3 \\ 7 \\ 0
        \end{array}
        \mright] = \begin{bmatrix}
            29
        \end{bmatrix}\]

        \[ A_{21}B_{11} = \renewcommand\arraystretch{1.3}
        \mleft[ 
        \begin{array}{cccc}
            1 & 5 & 6 & 1
        \end{array}
        \mright] \renewcommand\arraystretch{1.3}
        \mleft[
        \begin{array}{c}
            1 \\ 5 \\ -1 \\ 3
        \end{array}
        \mright] = \begin{bmatrix}
            23
        \end{bmatrix}\]

        \[ A_{21}B_{11} = \renewcommand\arraystretch{1.3}
        \mleft[ 
        \begin{array}{cccc}
            1 & 5 & 6 & 1
        \end{array}
        \mright] \renewcommand\arraystretch{1.3}
        \mleft[
        \begin{array}{c}
            4 \\ 2 \\ 5 \\ -3
        \end{array}
        \mright] = \begin{bmatrix}
            41
        \end{bmatrix}\]

        Then combining our results, \[ 
        AB = \renewcommand\arraystretch{1.3}
        \mleft[ 
        \begin{array}{c|c|c}
            \begin{bmatrix} -1 \\ 37 \end{bmatrix} & \begin{bmatrix} 23 \\ -13 \end{bmatrix} & \begin{bmatrix} -10 \\ 8 \end{bmatrix} \\ 
            \hline
            [29] & [23] & [41]
        \end{array}
        \mright]  = \begin{bmatrix}
            -1 & 23 & -10 \\ 
            37 & -13 & 8 \\ 
            29 & 23 & 41
        \end{bmatrix}
        \]
    \end{solution}

    \question
    \textbf{18 (a)} Show that if $A$ has a row of zeroes and $B$ is any matrix for which $AB$ is defined, then $AB$ also has a row of zeroes. 
    \begin{solution}
        
        Let $A = \begin{bmatrix}
            0 & 0 & ... & 0 \\ 
            a_{21} & a_{22} & ... & a_{2n} \\ 
            \vdots & \vdots & & \vdots \\
            a_{m1} & a_{m2} & ... & a_{mn} 
        \end{bmatrix}$ and $ B = \begin{bmatrix}
            b_{11} & b_{12} & ... & b_{1k} \\ 
            b_{21} & b_{22} & ... & b_{2k} \\ 
            \vdots & \vdots & & \vdots \\ 
            b_{n1} & b_{n2} & ... & b_{nk}
        \end{bmatrix} $.

        Then \\ $ AB = \begin{bmatrix}
            0b_{11} + 0b_{21} + ... + 0b_{n1} & 0b_{12} + 0b_{22} + ... + 0b_{n2} & ... & 0b_{1k} + 0b_{2k} + ... +0b_{nk} \\ 
            a_{21}b_{11} + a_22b_{21} ... +a_{2n}b_{n1} & a_{21}b_{12} + a_22b_{22} ... +a_{2n}b_{n2} & ... & a_{21}b_{1k} + a_22b_{2k} ... +a_{2n}b_{nk}  \\ 
            \vdots & \vdots &  & \vdots \\ 
            a_{m1}b_{11} + a_{m2}b_{21} + ... +a_{mn}b_{n1} & a_{m1}b_{12} + a_{m2}b_{22} + ... +a_{mn}b_{n2} & & a_{m1}b_{1k} + a_{m2}b_{2k} + ... +a_{mn}b_{nk}  
        \end{bmatrix} $ \\ 
        $ AB = \begin{bmatrix}
            0 & 0 & ... & 0 \\ 
            \vdots & \vdots & & \vdots
        \end{bmatrix} $

        Hence shown that if $A$ has a row of zeroes and $B$ is any matrix for which $AB$ is defined, then $AB$ also has a row of zeroes. The position of the row of zeroes would not matter in $A$, since the row of zeroes would occur at the same position in $AB$.  
    \end{solution}

\end{questions}
\end{sloppypar}
\end{document}
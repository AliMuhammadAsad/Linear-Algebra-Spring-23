\documentclass[addpoints]{exam}

\usepackage{mathtools}
\usepackage{amsmath}
\usepackage{amssymb}
\usepackage{geometry}
\usepackage{venndiagram}
\usepackage{graphicx}
\usepackage{arydshln}
\makeatletter
  \renewcommand*\env@matrix[1][*\c@MaxMatrixCols c]{%
    \hskip -\arraycolsep
    \let\@ifnextchar\new@ifnextchar
  \array{#1}}
\makeatother

% Header and footer.
\pagestyle{headandfoot}
\runningheadrule
\runningfootrule
\runningheader{Assignment 2}{Linear Algebra}{}
\runningfooter{}{Page \thepage\ of \numpages}{}
\firstpageheader{}{}{}

\boxedpoints
\printanswers
\qformat{} %Comment this to number questions, uncomment this to not number questions

\newcommand\union\cup
\newcommand\inter\cap

\title{Linear Algebra Spring 23\\ Assignment 2 | Lecture 5}
\author{Ali Muhammad Asad}

\begin{document}
\maketitle
\begin{sloppypar}
\section*{\textbf{Chapter 1 : Linear Equations and Matrices}}
% \subsection*{\textbf{Ex Set 1.4 : Inverses; Rules of Matrix Arithmetic}}
% \vspace{1mm}
\begin{questions}
    \question
    \textbf{Q1. } (a) Under what conditions $ AB = BA $ where $ A = \begin{bmatrix}
        a_{11} & a_{12} \\ a_{21} & a_{22}
    \end{bmatrix} $, $ B = \begin{bmatrix}
        b_{11} & b_{12} \\ b_{21} & b_{22}
    \end{bmatrix} $ 

    \hspace{8.2mm} (b) If $A$ is a matrix then $ A^rA^s = A^{r + s}  \; [\forall r, s \in \mathbb{Z^+}]$. 
    
    \hspace{14mm} Is this result true for negative integers also? Justify your answer

    \hspace{8.2mm} (c) If $ A = \begin{bmatrix}
        0 & 1 \\ 0 & 2
    \end{bmatrix}, \; B = \begin{bmatrix}
        1 & 1 \\ 3 & 4
    \end{bmatrix} $ and $ c = \begin{bmatrix}
        2 & 5 \\ 3 & 4
    \end{bmatrix} $. Then $ AB = BC = \begin{bmatrix}
        3 & 4 \\ 6 & 8
    \end{bmatrix} $ but $ B \neq C $. 
    
    \hspace{14mm} Why?
    \begin{solution}
        \begin{parts}
            \part For $AB = BA$: 
            $$ \begin{bmatrix}
                a_{11} & a_{12} \\ a_{21} & a_{22}
            \end{bmatrix}\begin{bmatrix}
                b_{11} & b_{12} \\ b_{21} & b_{22}
            \end{bmatrix} = \begin{bmatrix}
                b_{11} & b_{12} \\ b_{21} & b_{22}
            \end{bmatrix}\begin{bmatrix}
                a_{11} & a_{12} \\ a_{21} & a_{22}
            \end{bmatrix} $$
            $$ 
            \begin{bmatrix}
                a_{11}b_{11} + a_{12}b_{21} & a_{11}b_{12} + a_{12}b_{22} \\ 
                a_{21}b_{11} + a_{22}b_{21} & a_{21}b_{12} + a_{22}b_{22}
            \end{bmatrix} = \begin{bmatrix}
                b_{11}a_{11} + b_{12}a_{21} & b_{11}a_{12} + b_{12}a_{22} \\ 
                b_{21}a_{11} + b_{22}a_{21} & b_{21}a_{12} + b_{22}a_{22}
            \end{bmatrix} $$
            Therefore, by comparison: 
            
            $ a_{11}b_{11} + a_{12}b_{21} = b_{11}a_{11} + b_{12}a_{21}  \\ 
            a_{11}b_{12} + a_{12}b_{22} = b_{11}a_{12} + b_{12}a_{22} \\ 
            a_{21}b_{11} + a_{22}b_{21} = b_{21}a_{11} + b_{22}a_{21} \\ 
            a_{21}b_{12} + a_{22}b_{22} = b_{21}a_{12} + b_{22}a_{22}
            $ 

            So: 
            
            $ a_{12}b_{21} = b_{12}a_{21} \\ 
            a_{11}b_{12} + a_{12}b_{22} = b_{11}a_{12} + b_{12}a_{22} \\ 
            a_{21}b_{11} + a_{22}b_{21} = b_{21}a_{11} + b_{22}a_{21} \\
            a_{21}b_{12} = b_{21}a_{12}
            $ 

            \part Let $ A = \begin{bmatrix}
                1 & 1 \\ 0 & 0
            \end{bmatrix}, $ $ r = 1, s = -1 $

            Then $ A^{1-1} = A^1 A^{-1} $ \\ 
            $ \implies A^0 = A^1 A^{-1} $ \\ 
            $ \implies I = A^1 A^{-1} $ 

            Then, $ A^{-1} = \begin{bmatrix}
                1 & 1 \\ 0 & 0
            \end{bmatrix}^{-1} $, however, $ A^{-1} $ does not exist as $A$ has a row of zeroes, so determinant of $A$ is zero. Hence we can conclude that it is not valid for negative numbers.

            \part Since $A$ has a column of zeroes, then $A^{-1}$ does not exist. Hence, $ AB = AC \implies A^{-1}AB = A^{-1}AC \implies IB = IC \implies B = C $ does not hold as $ A^{-1} $ does not exist. Therefore, $ B \neq C $.
        \end{parts}
    \end{solution}

    \question
    \textbf{Q2. } Using the technique of forming a block matrix $\begin{bmatrix}[c:c]
        A & I
    \end{bmatrix}$ and performing EROS such that $ \begin{bmatrix}[c:c]
        A & I
    \end{bmatrix} \xrightarrow{EROS} \begin{bmatrix}[c:c]
        I & A^{-1}
    \end{bmatrix}$. Find the inverse of the following where $A$ is given by \\ (a) $ \begin{bmatrix}
        1 & 0 & 2 \\ 2 & -1 & 3 \\ 4 & 1 & 8
    \end{bmatrix} $ (b) $ \begin{bmatrix}
        1 & 2 & -4 \\ -1 & -1 & 5 \\ 2 & 7 & -3
    \end{bmatrix} $
    \begin{solution}
        $ \begin{bmatrix}[ccc : ccc]
            1 & 0 & 2 & 1 & 0 & 0 \\ 
            2 & -1 & 3 & 0 & 1 & 0 \\ 
            4 & 1 & 8 & 0 & 0 & 1
        \end{bmatrix} $ \\ 
        R$_2 - 2$R$_1$, R$_3 - 4$R$_1$ $ \begin{bmatrix}[ccc:ccc]
            1 & 0 & 2 & 1 & 0 & 0 \\ 
            0 & -1 & -1 & -2 & 1 & 0 \\ 
            0 & 1 & 0 & -4 & 0 & 1 
        \end{bmatrix} $ \hspace*{5mm} R$_2 \times -1$ $ \begin{bmatrix}[ccc:ccc]
            1 & 0 & 2 & 1 & 0 & 0 \\ 
            0 & 1 & 1 & 2 & -1 & 0 \\ 0 & 1 & 0 & -4 & 0 & 1
        \end{bmatrix} $

        R$_2 - $R$_3$ $ \begin{bmatrix}[ccc:ccc]
            1 & 0 & 2 & 1 & 0 & 0 \\ 0 & 0 & 1 & 6 & -1 & -1 \\ 0 & 1 & 0 & -4 & 0 & 1             
        \end{bmatrix} $ \hspace*{5mm} R$_3 \rightleftarrows  $R$_2$ $ \begin{bmatrix}[ccc:ccc]
            1 & 0 & 2 & 1 & 0 & 0 \\ 0 & 1 & 0 & -4 & 0 & 1 \\ 0 & 0 & 1 & 6 & -1 & -1            
        \end{bmatrix} $ 

        R$_1 -2$R$_3$ $ \begin{bmatrix}[ccc:ccc]
            1 & 0 & 0 & -11 & 2 & 2 \\ 0 & 1 & 0 & -4 & 0 & 1 \\ 0 & 0 & 1 & 6 & -1 & -1            
        \end{bmatrix} $

        (b) follows just like the above part
    \end{solution}
    \pagebreak
    \question
    \textbf{Q3. } Solve the following system of equations by reducing them to Echelon form (Guassian Elimination method) \\ 
    $ x + y + 2z = 9 \\ 2x + 4y - 3z = 1 \\ 3x + 6y -5z = 0 $
    \begin{solution}
        $ \begin{bmatrix}[ccc | c]
            1 & 1 & 2 & 9 \\ 2 & 4 & -3 & 1 \\ 3 & 6 & -5 & 0            
        \end{bmatrix} $

        we follow the same steps as the above question to get to the Identity matrix at the left hand side of the line, then at the right hand side we get the solution which is: \\ 
        $ \begin{bmatrix}[ccc | c]
            1 & 0 & 0 & 1 \\ 0 & 1 & 0 & 2 \\ 0 & 0 & 0 & 3            
        \end{bmatrix} $        
    \end{solution}

    \question
    \textbf{Q4. } Solve the following system by Gauss-Jordan elimination (reduced row Echelon form) \\ 
    $ 2x_1 + 2x_2 + 2x_3 = 0 \\ -2x_1 + 5x_2 + 2x_3 = 1 \\ 8x_1 + x_2 + 4x_3 = -1 $
    \begin{solution}
        Same method as the above part
    \end{solution}

    \question
    \textbf{Q5. } Reduce $ \begin{bmatrix}
        2 & 1 & 3 \\ 0 & -2 & 7 \\ 3 & 4 & 5
    \end{bmatrix} $ to reduced row echelon form without introducing any fractions.
    \begin{solution}
        Same as above part
    \end{solution}
    \pagebreak
    \question
    \textbf{Q6. } Find two different row Echelon forms of $ \begin{bmatrix}
        1 & 3 \\ 2 & 7
    \end{bmatrix} $ 
    \begin{solution}
        
        One possible solution is: \\ 
        R$_2 - 2$R$_1$ $ \begin{bmatrix}
            1 & 3 \\ 0 & 1
        \end{bmatrix} $

        Another solution could be: \\ 
        $ \displaystyle\frac{\text{R$_2$}}{2} $ $ \begin{bmatrix}
            1 & 3 \\ 1 & \frac{7}{2}
        \end{bmatrix} $ \hspace*{5mm} R$_2 - $R$_1  \begin{bmatrix}
            1 & 3 \\ 0 & \frac{1}{2}
        \end{bmatrix}$
    \end{solution}

    \question
    \textbf{Q7. } Exercise set 1.3 Q25
    \begin{solution}
        Done in Homework 3 i
    \end{solution}

    \question
    \textbf{Q8. } Exercise set 1.3 Q18 and 19
    \begin{solution}
        Done in Homework 3 i
    \end{solution}

\end{questions}
\end{sloppypar}
\end{document}